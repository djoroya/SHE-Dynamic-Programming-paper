
%\documentclass{elsart}               % The use of LaTeX2e is preferred.

\documentclass[twocolumn]{autart}    % Enable this line and disable the 
                                     % preceding line to obtain a two-column 
                                     % document whose style resembles the
                                     % printed Automatica style.


\usepackage{graphicx}          % Include this line if your 
                               % document contains figures,
%\usepackage[dvips]{epsfig}    % or this line, depending on which
                               % you prefer.

\usepackage{graphicx}

\usepackage{bm}

\usepackage{amsmath}
\usepackage{amssymb}
\usepackage{ntheorem}

\usepackage[colorlinks=true,citecolor=blue]{hyperref}
\usepackage{breakcites}


\edef\endfrontmatter{%
	\unexpanded\expandafter{\endfrontmatter}% the current code
	\noexpand\endNoHyper % add \endNoHyper at the end to match \NoHyper
}

\newtheorem{problem}{Problem}[section]
\newtheorem{definition}{Definition}[section]
\newtheorem{remark}{Remark}[section]
\newtheorem{theorem}{Theorem}[section]
\newtheorem{proposition}{Proposition}[section]
\newtheorem{lemma}{Lemma}[section]
\newtheorem{simulation}{Simulation}[section]

\theoremstyle{plain}
\newtheorem*{proof}{\textit{Proof:}}

\numberwithin{equation}{section}

\DeclareMathOperator*{\argmin}{arg\,min}
\usepackage{soul}

\usepackage{xcolor}

\begin{document}

\begin{frontmatter}
\runtitle{Selective Harmonic Elimination via Optimal Control}   % Running title for regular 
                                              % papers but only if the title  
                                              % is over 5 words. Running title 
                                              % is not shown in output.

\title{Caracterization of Space of Solution for selective Harmonic elemination problem} % Title, preferably not more 
                                                 % than 10 words.

\thanks[footnoteinfo]{This paper was not presented at any IFAC 
meeting.}

\author[UAM,FD]{Carlos Esteve}\ead{carlos.esteve@uam.es},               % e-mail address 
\author[UD]{Deyviss Jes\'us Oroya}\ead{djoroya@deusto.es}  % (ead) as shown
\address[FD]{Chair of Computational Mathematics, Fundaci\'on Deusto, Avenida de las Universidades 24, 48007 Bilbao, Basque Country, Spain.}  %
\address[UD]{Universidad de Deusto, Avenida de las Universidades 24, 48007 Bilbao, Basque Country, Spain.}  %
\address[UAM]{Departamento de Matem\'aticas, Universidad Aut\'onoma de Madrid, 28049 Madrid, Spain.}  % Please supply                                              
          
\begin{keyword}                           % Five to ten keywords,  
Selective Harmonic Elimination; Finite Set Control, Piecewise Linear function.               % chosen from the IFAC 
\end{keyword}                             % keyword list or with the 
                                          % help of the Automatica 
                                          % keyword wizard


\begin{abstract}                          % Abstract of not more than 200 words.
En este documento caracterizaremos el espacio de soluciones dado un problema selective harmonic eleimination. Para ello utilizaremos la formulación como problema de control óptimo de SHE, para luego encontar el conjunto de controlabilidad, de manera que en la formulación SHE se traduce a la existencia de solución.
\end{abstract}

\end{frontmatter} 


\section{Introduction and motivations}\label{Section1}

Selective Harmonic Elimination (SHE) \cite{Rodriguez2002} is a well-known methodology in electrical engineering, employed to improve the performances of a converter by controlling the phase and amplitude of the harmonics in its output voltage. As a matter of fact, this technique allows to increase the power of the converter and, at the same time, to reduce its losses. 
%


This formulation was introduce in [nuestro paper], In what follows, with the notation $\mathcal U$ we will always refer to a finite set of real numbers, contained in the interval $[-1,1]$
\begin{align}\label{eq:Udef}
	\mathcal U = \{u_\ell\}_{\ell=1}^L\subset [-1,1],
\end{align}
with cardinality $|\mathcal U| = L$. 

\vspace{1em}
\begin{problem}\label{pb:SHEpControl}
    Let $\mathcal{U}$ be defined as in \eqref{eq:Udef} and let $\mathcal{E}_a = \{e_a^i\}_{i=1}^{N_a}$ and $\mathcal{E}_a = \{e_b^j\}_{j=1}^{N_b}$ be two sets of odd numbers. Given the vectors $\bm{a}_T \in \mathbb{R}^{N_a}$ and $\bm{b}_T \in \mathbb{R}^{N_b} $, let us define $\bm{x}_0=[\bm{a}_T,\bm{b}_T]^\top \in \mathbb{R}^{N_a}\times\mathbb{R}^{N_b}$. We look for $u:\in [0,\pi)\to\mathcal{U}$ such that the solution of 
    \begin{equation}\label{eq:CauchyReversed}
        \begin{cases}
            \displaystyle\dot{\bm{x}}(\tau) = -\frac 2\pi\bm{\mathcal{D}}(\tau)u(\tau),  & \tau \in [0,\pi)
            \\[6pt]
            \bm{x}(0) = \bm{x}_0
        \end{cases},
    \end{equation}
    and which satisfies $\bm{x}(\pi)=0$.

    Where $\bm{x}(t) \in \mathbb{R}^{N_a + N_b}$ and 
    \begin{gather}
        \bm{\mathcal{D}}(\tau) = 
        \begin{bmatrix} \bm{\mathcal{D}}^\alpha(\tau) \ \bm{\mathcal{D}}^\beta(\tau) \end{bmatrix}^T    
    \end{gather}
    and where $\bm{\mathcal{D}}^\beta(\tau) \in \mathbb{R}^{N_a} $ and $ \bm{\mathcal{D}}^\beta(\tau) \in \mathbb{R}^{N_b}$ are,
    \begin{gather}\label{eq:DalphaDbeta}
        \begin{align}
            \bm{\mathcal{D}}^\alpha(\tau) = 
            \begin{bmatrix} 
                \cos(e_a^1\tau) \\ \vdots \\ \cos(e_a^{N_a}\tau) 
            \end{bmatrix},
            \quad \bm{\mathcal{D}}^\beta(\tau) = 
            \begin{bmatrix} 
                \sin(e_b^1\tau) \\ \vdots \\ \sin(e_b^{N_b}\tau) 
            \end{bmatrix} 
        \end{align} 
    \end{gather}
    \end{problem}
    Nos preguntamos cúal es el conjunto controlable $\mathcal{R}$ de Problema \ref{pb:SHEpControl}, es decir para que $\bm{x}_0 \in \mathbb{R}^{N_a+N_b}$ existe un control que sea capaz de llevar el sistema dinámico al origen.

    Suponiendo que en el instante $\tau = \pi$, el sistema se encuentra en $\bm{x}(\pi) = \bm{0}$, nos podemos preguntar que posible acción nos podría haber llevado a este punto. Es decir en un instante $\tau = \pi - \Delta \tau$, dado que solo podemos tomar un conjunto de acciones $\mathcal{U}$, suponiendo que en todo el intervalo $\Delta t$ la derivada del sistema se mantiene constante, entonces podemos afirmar que los puntos más lejanos de los cuales podemos provenier son 
    \begin{gather}
        x(\pi - \Delta \tau) =   
        \begin{cases}
            +\frac{2}{\pi} \bm{\mathcal{D}}(\pi - \Delta \tau)\\[5pt]
            -\frac{2}{\pi} \bm{\mathcal{D}}(\pi - \Delta \tau) 
        \end{cases}
    \end{gather}
    Considerando el instante de timepo anterior $\tau = \tau - 2\Delta \tau$, podremos razonar de la misma manera, pero esta vez existen ya dos puntos a donde podemos llegar para que luego el sistema sea controlable. Entonces los puntos controlables serían 
    \begin{gather*}
        x(\pi - 2\Delta \tau) = 
        \begin{cases}
            \frac{2}{\pi} \Big( 
            +\bm{\mathcal{D}}(\pi - \Delta \tau) + \bm{\mathcal{D}}(\pi - 2\Delta \tau) \Big) \\
            \frac{2}{\pi} \Big( 
            -\bm{\mathcal{D}}(\pi - \Delta \tau) + \bm{\mathcal{D}}(\pi - 2\Delta \tau) \Big) \\
            \frac{2}{\pi} \Big( 
            +\bm{\mathcal{D}}(\pi - \Delta \tau) - \bm{\mathcal{D}}(\pi - 2\Delta \tau) \Big) \\
            \frac{2}{\pi} \Big( 
            -\bm{\mathcal{D}}(\pi - \Delta \tau) - \bm{\mathcal{D}}(\pi - 2\Delta \tau) \Big) \\
        \end{cases}
    \end{gather*}

    Viendo el ejemplo anterior, vemos que en un instante de tiempo anterior $\Delta \tau$ el tamaño de conjunto de controlabilidad es igual a dos puntos, en el intante anterior $2\Delta \tau $ se añaden cuatro puntos más.

    Solo necestiamos analizar el borde del conjunto de controlabilidad, dado que por el Teorema XX [], este tipo de conjunto es convexo y compacto.  
\input{contens/P002-Model}
\section{Proyection of the real system in direction $w$}
Consideramos el sistema de control proyectado en la dirección $\bm{w} \in \mathbb{R}^{N_b}  / \  ||\bm{w}|| = 1$, cuyo estado llamaremos $x^w(\tau) = \bm{w} \cdot\bm{x}(\tau) \in \mathbb{R}$
\begin{gather}
    \begin{cases}
        \displaystyle\dot{x}^w(\tau) = 
        -\frac{2}{\pi} 
        \big( \bm{\mathcal{D}}(\tau) \cdot \bm{\omega} \big)  
        u(\tau) 
        & \tau \in [0,\pi]\\[6pt]
        \displaystyle x^w(0) =  [\bm{x}_0 \cdot \bm{w}]
    \end{cases}
\end{gather}
Para esta dinámica proyectada existirá un valor maximo  $\bm{x}_{m}^w$ para la condición inicial para el cual exista un control $u(\tau)$ que pueda conducir el sistema al origen en tiempo $\tau = \pi/2$.  

 
Para hallar el valor $\bm{x}_{m}^w$ podemos pensar el cómo debe ser la estrategia de control para que el estado recorra la máxima distancia posible. 
%
La dinámica proyectada es lo más grande posible cuando la derivada toma siempre valores negativos. 
% 
Dado que podemos elegir $u(\tau)$ en el intervalo $[-1,1]$ la elegiremos de tal manera que la derivada temporal sea siempre negativa y lo más grande posible. 
%
Entonces la ecuación diferencial del punto más lejano en la direncción $\bm{w}$ en un instante de tiempo $\tau$ lo denotamos como $b^{\bm{w}}$ que podemos alcanzar en $\tau = \pi$ es:
\begin{gather}
    \begin{cases}
        \displaystyle\dot{b}^w(\tau) = 
        - \frac{2}{\pi} 
        \big| \bm{\mathcal{D}}(\tau) \cdot \bm{\omega} \big|
        & \tau \in [0,\pi]\\[6pt]
        \displaystyle b^w(\pi) =  0
    \end{cases}
\end{gather}
Con condición final el origen de coordenadas. También podemos escribirlo en  su versión integral:
\begin{gather}
   b^{\bm{w}}(\tau) = \frac{2}{\pi} \int_\tau^{\pi}
    \big| \bm{\mathcal{D}}(\eta) \cdot \bm{\omega} \big| d\eta
\end{gather}
Entonces podemos integralo:
\begin{gather}\label{eq:xmax}
   b^{\bm{w}}(\tau) =  
    \frac{2}{\pi} \Big|  \mathcal{G}_{\bm{w}}(\eta) \
    \text{sign}(\bm{\mathcal{D}}(\eta) \cdot \bm{w})\Big|_{\eta=\tau}^{\eta=\pi}
\end{gather}
Donde hemos utilzado $\frac{d}{d\tau} \mathcal{G}_{\bm{w}}(\tau) =  \bm{\mathcal{D}}
(\tau) \cdot \bm{w}$
\begin{gather}
    \bm{\mathcal{D}}(\tau) \cdot \bm{w} = 
    \sum_{i \in \mathcal{E}_a}  \cos(i\tau)w^\alpha_i+ 
    \sum_{j \in \mathcal{E}_b}  \sin(j\tau)w^\beta_j
\end{gather}
La primitiva $\mathcal{G}_{\bm{w}}(\tau)$ que buscamos es de la forma: 
\begin{gather}
    \mathcal{G}_{\bm{w}}(\tau) =  
    \sum_{i \in \mathcal{E}_a}  \sin(i\tau)\frac{w^\alpha_i}{i} - 
    \sum_{j \in \mathcal{E}_b}  \cos(j\tau)\frac{w^\beta_j}{j}
\end{gather}
Entonces la Ecuación \eqref{eq:xmax} queda:
\begin{gather}
   b^{\bm{w}}(\tau) =  b^{\bm{w}}(\pi) - \frac{2}{\pi} \mathcal{G}_{\bm{w}}(\tau) \ \text{sign}(\bm{\mathcal{D}}(\tau) \cdot \bm{w})
\end{gather}
Ahora calculamos $b(\pi)$: 
\begin{gather}
    \bm{\mathcal{D}}(\pi) \cdot \bm{w} = 
    - \sum_{i \in \mathcal{E}_a} w_j^\alpha   \\
    \mathcal{G}_{\bm{w}}(\pi) = \sum_{j \in \mathcal{E}_b} \frac{w_j^\beta}{j}
\end{gather}
Entonces 
\begin{gather}
    x_{m}^w(\pi) = - \frac{2}{\pi} \sum_{j \in \mathcal{E}_b} \frac{w_j^\beta}{j}
\end{gather}
\begin{theorem}
Entonces dados una trayectoria $\bm{\beta}(\tau)  |  \tau \in [0,\pi/2]$ generada mediante una estrategia $u(\tau)$ y una dirección $\bm{w}$, si la proyección  $x^w(\tau) = (\bm{x}(\tau),\bm{w})$ es siempre tal que $ x^w(\tau) < x^{\bm{w}}_{max }(\tau) \ |  \forall \tau \in [0,\pi]$ entonces $x^w(\pi)\leq 0$. Es decir la proyección de $\bm{x}(\tau)$ puede alcanzar el origen de coordenadas en tiempo $\tau = \pi$ e incluso ir más alla del origen.
\end{theorem}

Final 
\begin{gather}
   b^{\bm{w}}(\tau) =  
    x^{\bm{w}}_m(\pi)- \frac{2}{\pi} 
    \Big[ 
    \mathcal{G}_{\bm{w}}(\tau) \ \text{sign}(\bm{\mathcal{D}}(\tau) \cdot \bm{w})
    \Big]
\end{gather}


\begin{gather}
    \mathcal{P}_{\bm{w}} =
    \{ \bm{x}(t) \in \mathbb{R}^{N_a + N_b}|
     \bm{w} \cdot \bm{x}(t) + x^{\bm{w}}_m(\tau) = 0\}
\end{gather}
\section{Obtención de estrategias $u(\tau)$}

Dado una condición inicial $\bm{x}(\tau)$, como podríamos decir si es controlable  o no?    
\input{contens/P004-HJB-related}
\section{Numerical Simulations}
\section{Conclusions}
\appendix

\section{Primitiva}

\begin{gather}
    \int |f(x)|dx = -F(x)sign(f(x)) \\
    \frac{dF(x)}{dx} = f(x) 
\end{gather}

clickar en el código latex par entrar en el link de wolframalpha
\url{https://www.wolframalpha.com/input/?i=Int+abs%28+a+sin%28+j+x+%29+++%2B+b+sin%28+k+x%29%29}
\begin{ack}                               % Place acknowledgements
This work has been partially funded by the European Research Council (ERC) under the European Union’s Horizon 2020 research and innovation program
(grant agreement No. 694126-DyCon).
\end{ack}
 
\bibliographystyle{apalike}        % Include this if you use bibtex 
\bibliography{bib}           % and a bib file to produce the 
                                 % bibliography (preferred). The
                                 % correct style is generated by
                                 % Elsevier at the time of printing.


 

\end{document} 