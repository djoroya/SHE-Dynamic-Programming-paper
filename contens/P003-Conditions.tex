\section{Proyection of the real system in direction $w$}
Consideramos el sistema de control proyectado en la dirección $\bm{w} \in \mathbb{R}^{N_b}  / \  ||\bm{w}|| = 1$, cuyo estado llamaremos $x^w(\tau) = \bm{w} \cdot\bm{x}(\tau) \in \mathbb{R}$
\begin{gather}
    \begin{cases}
        \displaystyle\dot{x}^w(\tau) = 
        -\frac{2}{\pi} 
        \big( \bm{\mathcal{D}}(\tau) \cdot \bm{\omega} \big)  
        u(\tau) 
        & \tau \in [0,\pi]\\[6pt]
        \displaystyle x^w(0) =  [\bm{x}_0 \cdot \bm{w}]
    \end{cases}
\end{gather}
Para esta dinámica proyectada existirá un valor maximo  $\bm{x}_{m}^w$ para la condición inicial para el cual exista un control $u(\tau)$ que pueda conducir el sistema al origen en tiempo $\tau = \pi/2$.  

 
Para hallar el valor $\bm{x}_{m}^w$ podemos pensar el cómo debe ser la estrategia de control para que el estado recorra la máxima distancia posible. 
%
La dinámica proyectada es lo más grande posible cuando la derivada toma siempre valores negativos. 
% 
Dado que podemos elegir $u(\tau)$ en el intervalo $[-1,1]$ la elegiremos de tal manera que la derivada temporal sea siempre negativa y lo más grande posible. 
%
Entonces la ecuación diferencial del punto más lejano en la direncción $\bm{w}$ en un instante de tiempo $\tau$ lo denotamos como $b^{\bm{w}}$ que podemos alcanzar en $\tau = \pi$ es:
\begin{gather}
    \begin{cases}
        \displaystyle\dot{b}^w(\tau) = 
        - \frac{2}{\pi} 
        \big| \bm{\mathcal{D}}(\tau) \cdot \bm{\omega} \big|
        & \tau \in [0,\pi]\\[6pt]
        \displaystyle b^w(\pi) =  0
    \end{cases}
\end{gather}
Con condición final el origen de coordenadas. También podemos escribirlo en  su versión integral:
\begin{gather}
   b^{\bm{w}}(\tau) = \frac{2}{\pi} \int_\tau^{\pi}
    \big| \bm{\mathcal{D}}(\eta) \cdot \bm{\omega} \big| d\eta
\end{gather}
Entonces podemos integralo:
\begin{gather}\label{eq:xmax}
   b^{\bm{w}}(\tau) =  
    \frac{2}{\pi} \Big|  \mathcal{G}_{\bm{w}}(\eta) \
    \text{sign}(\bm{\mathcal{D}}(\eta) \cdot \bm{w})\Big|_{\eta=\tau}^{\eta=\pi}
\end{gather}
Donde hemos utilzado $\frac{d}{d\tau} \mathcal{G}_{\bm{w}}(\tau) =  \bm{\mathcal{D}}
(\tau) \cdot \bm{w}$
\begin{gather}
    \bm{\mathcal{D}}(\tau) \cdot \bm{w} = 
    \sum_{i \in \mathcal{E}_a}  \cos(i\tau)w^\alpha_i+ 
    \sum_{j \in \mathcal{E}_b}  \sin(j\tau)w^\beta_j
\end{gather}
La primitiva $\mathcal{G}_{\bm{w}}(\tau)$ que buscamos es de la forma: 
\begin{gather}
    \mathcal{G}_{\bm{w}}(\tau) =  
    \sum_{i \in \mathcal{E}_a}  \sin(i\tau)\frac{w^\alpha_i}{i} - 
    \sum_{j \in \mathcal{E}_b}  \cos(j\tau)\frac{w^\beta_j}{j}
\end{gather}
Entonces la Ecuación \eqref{eq:xmax} queda:
\begin{gather}
   b^{\bm{w}}(\tau) =  b^{\bm{w}}(\pi) - \frac{2}{\pi} \mathcal{G}_{\bm{w}}(\tau) \ \text{sign}(\bm{\mathcal{D}}(\tau) \cdot \bm{w})
\end{gather}
Ahora calculamos $b(\pi)$: 
\begin{gather}
    \bm{\mathcal{D}}(\pi) \cdot \bm{w} = 
    - \sum_{i \in \mathcal{E}_a} w_j^\alpha   \\
    \mathcal{G}_{\bm{w}}(\pi) = \sum_{j \in \mathcal{E}_b} \frac{w_j^\beta}{j}
\end{gather}
Entonces 
\begin{gather}
    x_{m}^w(\pi) = - \frac{2}{\pi} \sum_{j \in \mathcal{E}_b} \frac{w_j^\beta}{j}
\end{gather}
\begin{theorem}
Entonces dados una trayectoria $\bm{\beta}(\tau)  |  \tau \in [0,\pi/2]$ generada mediante una estrategia $u(\tau)$ y una dirección $\bm{w}$, si la proyección  $x^w(\tau) = (\bm{x}(\tau),\bm{w})$ es siempre tal que $ x^w(\tau) < x^{\bm{w}}_{max }(\tau) \ |  \forall \tau \in [0,\pi]$ entonces $x^w(\pi)\leq 0$. Es decir la proyección de $\bm{x}(\tau)$ puede alcanzar el origen de coordenadas en tiempo $\tau = \pi$ e incluso ir más alla del origen.
\end{theorem}

Final 
\begin{gather}
   b^{\bm{w}}(\tau) =  
    x^{\bm{w}}_m(\pi)- \frac{2}{\pi} 
    \Big[ 
    \mathcal{G}_{\bm{w}}(\tau) \ \text{sign}(\bm{\mathcal{D}}(\tau) \cdot \bm{w})
    \Big]
\end{gather}


\begin{gather}
    \mathcal{P}_{\bm{w}} =
    \{ \bm{x}(t) \in \mathbb{R}^{N_a + N_b}|
     \bm{w} \cdot \bm{x}(t) + x^{\bm{w}}_m(\tau) = 0\}
\end{gather}
\section{Obtención de estrategias $u(\tau)$}

Dado una condición inicial $\bm{x}(\tau)$, como podríamos decir si es controlable  o no?    